\documentclass{pre-tfg}

% \showhelp  % comenta o borra para eliminar ayudas

\title{Nextinit, desarrollo de aplicaciones móviles con tecnologías híbridas}
\author{Eduardo Parra Mazuecos}
\advisorFirst{Antonios Chatzisavvas}
\advisorDepartment{DEPARTAMENTO DE TECNOLOGÍAS Y SISTEMAS DE INFORMACIÓN}
\advisorSecond{Manuel Ángel Serrano Martín}
\intensification{INGENIERÍA DEL SOFTWARE}
\docdate{2018}{Abril}


\begin{document}

    \maketitle
    \tableofcontents

    \newpage

    El anteproyecto recogería, en un \textcolor[rgb]{0.5,0.0,0.0}{máximo de 10 páginas},
    los siguientes apartados:

    \begin{itemize}
        \item Introducción (muy recomendable aunque no obligatorio)
        \item Tecnología específica cursada por el alumno
        \item Objetivos
        \item Método y fases de trabajo
        \item Medios que se pretenden utilizar
        \item Bibliografía básica consultada en la elaboración del anteproyecto
        \item Contrato de propiedad intelectual (si lo hubiera)
    \end{itemize}


    \section{INTRODUCCIÓN}

    \textcolor[rgb]{0.5,0.0,0.0}{TODO}
    El capítulo de introducción podrá abordar los siguientes aspectos:

    \begin{itemize}
        \item Introducción al tema, entorno en el que el trabajo desempeñará
        su objetivo, justificación de la importancia del trabajo abordado.
        \item Motivación y antecedentes (con algunas referencias bibliográficas).
        \item Descripción gráfica del proyecto (es aconsejable incorporar una figura que describa
        el trabajo a desarrollar y que mejore la comprensión del mismo).
    \end{itemize}


    \section{TECNOLOGÍA ESPECÍFICA / INTENSIFICACIÓN / ITINERARIO CURSADO POR EL ALUMNO}


    \begin{table}[hp]
        \centering
        \caption{Tecnología Específica cursada por el alumno}
        \label{tab:tec-especifica}

        \zebrarows{1}
        \begin{tabular}{p{0.1\linewidth}p{0.4\linewidth}}
            & \textbf{Marcar la tecnología cursada} \\
            \hline
            & Tecnologías de la Información \\
            & Computación \\
            X & Ingeniería del Software \\
            & Ingeniería de Computadores \\
            \hline
        \end{tabular}
    \end{table}


    \clearpage


    \begin{table}[hp]
        \centering
        \caption{Justificación de las competencias específicas abordadas en el TFG}
        \label{tab:competencias}

        \zebrarows{1}
        \begin{tabular}{p{0.5\linewidth}p{0.4\linewidth}}
            \textbf{Competencia} & \textbf{Justificación} \\
            \hline
            Capacidad para desarrollar, mantener y evaluar servicios y sistemas software que satisfagan
            todos los requisitos del usuario y se comporten de forma fiable y eficiente, sean asequibles
            de desarrollar y mantener y cumplan normas de calidad, aplicando las teorías, principios,
            métodos y prácticas de la Ingeniería del Software.
            & El software del proyecto se desarrollará usando el patrón Flux~\cite{FLUX} y se utilizarán los principios
            y buenas practicas de Ingeniería del Software, como la utilización de metodologías ágiles y
            tecnologías para la mejora de la calidad.\\
            Capacidad para valorar las necesidades del cliente y especificar los requisitos software para
            satisfacer estas necesidades, reconciliando objetivos en conflicto mediante la búsqueda de
            compromisos aceptables dentro de las limitaciones derivadas del coste, del tiempo, de la existencia
            de sistemas ya desarrollados y de las propias organizaciones.
            & El proyecto requiere de unos requisitos flexibles, que pueden ser modificados a lo largo del
            desarrollo. Los requisitos se recogerán y validarán en las reuniones periódicas de seguimiento
            del desarrollo con las personas responsables de la empresa.\\
            Capacidad de dar solución a problemas de integración en función de las estrategias, estándares
            y tecnologías disponibles.
            & La aplicación resultante debe conectarse con una API REST para almacenar y obtener información
            necesaria de la base de datos.\\
            Capacidad de identificar y analizar problemas y diseñar, desarrollar, implementar, verificar y
            documentar soluciones software sobre la base de un conocimiento adecuado de las teorías, modelos
            y técnicas actuales.
            & Se analizará el cumplimiento del desarrollo, de acuerdo a la metodología. Los problemas o
            decisiones que se tomen durante el trascurso del proyecto serán documentadas de tal forma que
            pueda entenderse lo ocurrido correctamente. \\
            Capacidad de identificar, evaluar y gestionar los riesgos potenciales asociados que pudieran presentarse.
            & \textcolor[rgb]{0.5,0.0,0.0}{TODO}\\
            Capacidad para diseñar soluciones apropiadas en uno o más dominios de aplicación utilizando métodos
            de la ingeniería del software que integren aspectos éticos, sociales, legales y económicos.
            & Se velará el correcto cumplimiento de la LOPD~\cite{LOPD}, gestionando correctamente los datos de los
            usuarios y de las empresas.\\
            \hline
        \end{tabular}
    \end{table}


    \section{OBJETIVOS}

    El objetivo principal de este TFG es la creación de una aplicación multiplataforma (NextInit) para móviles,
    desarrollada con tecnologías híbridas que permita con un único desarrollo funcionar en IOS y en Android,
    adaptándose de forma adecuada a cada sistema operativo. La aplicación móvil será creada desde cero, se
    integrará mediante una API REST con la aplicación web de NextInit. Se incluirá la funcionalidad disponible
    en la anterior aplicación móvil de NextInit, además, de otras funcionalidades nuevas.
    \newline\newline
    Para alcanzar el objetivo principal de este TFG, antes se deben cumplir los siguientes subobjetivos:
    \begin{enumerate}
        \item Aprendizaje del lenguaje de programación (javascript), del framework de desarrollo (React Native)
        y de las librerías necesarias (redux, thunk…)
        \item Análisis y elicitación de los requisitos
        \item Estudio de la API de NextInit y en caso de necesitar alguna modificación para el correcto
        funcionamiento de la aplicación móvil, solicitarla.
        \item Configuración de un entorno de integración continua
        \item Diseño, implementación y pruebas para dispositivos IOS y Android
        \item Despliegue de la solución
    \end{enumerate}


    \section{MÉTODO Y FASES DE TRABAJO}

    Para el desarrollo se usará una metodología basada en eXtreme Programming~\cite{XP} adaptada para
    adecuarla a un equipo con un solo desarrollador, por lo que no habrá Pair Programming.
    Se ha elegido esta metodología porque es adecuada para un equipo pequeño y con requisitos
    muy cambiantes.

    Principios fundamentales de eXtreme Programming:
    \begin{enumerate}
        \item Planificación incremental - La planificación es iterativa, el cliente elige en al comienzo de cada
        iteración que historias de usuario se van a implementar.
        \item Testing - Los test son esenciales, es necesario cerar test unitarios y de aceptación. Los test se
        escriben antes de desarrollar el código.
        \item Refactorización -
    \end{enumerate}
    - Planificación incremental
    Testing
    Refactorización
    Diseño simple
    Integración continua
    Cliente en el equipo
    Releases pequeñas
    Semanas de 40 horas
    Estándares de codificación
    Unso de metaforas

    Ciclo de vida del proyecto:
    \begin{enumerate}
        \item Exploración -
    \end{enumerate}

    \textcolor[rgb]{0.5,0.0,0.0}{TODO}



    Para el desarrollo del proyecto, el alumno deberá seguir algún proceso o metodología afín
    al problema que pretende resolver. Para ello, deberá aportar una pequeña descripción del
    proceso o metodología (no más de una página) y \textbf{justificar su adecuación al
    problema a resolver}.

    Del mismo modo, el alumno podrá realizar una breve planificación de la ejecución del
    proyecto según el proceso o metodología seleccionada.

    Como parte de la descripción del método y las fases de trabajo, el alumno podrá incluir
    una descripción preliminar de las tareas, una planificación temporal, diagramas de Gantt o
    recursos similares que pueda considerar necesarios.

    Si hubiera más de una metodología que a juicio del alumno podría ser afín al proyecto,
    éstas deberán mencionarse, y justificar la que considera más adecuada (esto puede
    considerarse parte de la justificación a la adecuación al problema a resolver).


    \section{MEDIOS QUE SE PRETENDEN UTILIZAR}

    \subsection{Medios Hardware}

    Para la realización de este TFG se utilizará un ordenador, un móvil Android y un iPhone.
    El equipo necesario ha sido facilitado por Intelygenz.
    \begin{itemize}
        \item \textbf{MacBook Pro} Procesador Intel Core i5 @2.5GHz. 8GB RAM 1600 MHz. macOS High Sierra v10.13.3
        \item Android (Nexus 5x) \textcolor[rgb]{0.5,0.0,0.0}{TODO}
        \item iPhone (Iphone 7 o X) \textcolor[rgb]{0.5,0.0,0.0}{TODO}
    \end{itemize}

    \subsection{Medios Software}

    El software necesario para la realización del proyecto es el siguiente:
    \begin{itemize}
        \item Android Studio~\cite{ASTUDIO}
        \item ECMAScript~\cite{ECMA}
        \item Expo~\cite{Expo}
        \item IntelliJ IDEA Community~\cite{IDEA}
        \item React~\cite{REACT}
        \item React Native~\cite{RENA}
        \item Redux~\cite{REDUX}
        \item Redux Thunk~\cite{THUNK}
        \item Jest~\cite{JEST}
        \item Native Base~\cite{NABA}
        \item Standard~\cite{STAND}
        \item Xcode~\cite{XCODE}
    \end{itemize}


    \section{REFERENCIAS}

    \bibliographystyle{alpha}
    \singlespacing
    \bibliography{main}

    \section{CONTRATO DE PROPIEDAD INTELECTUAL}

    Proyecto desarrollado en base a una colaboración entre la empresa Desarrollos informáticos Intelygenz S.L. y
    la Escuela Superior de Informática (ESI) de Ciudad Real, bajo el convenio FORTE. Todo posible software desarrollado
    derivado de dicha colaboración será propiedad exclusivamente de Desarrollos informáticos Intelygenz S.L, pudiendo ser
    utilizado sin excepción alguna con fines comerciales. De igual manera, todo beneficio económico derivable de su venta,
    uso o explotación, será únicamente un derecho de Desarrollos informáticos Intelygenz S.L.

\end{document}


% Local Variables:
% coding: utf-8
% mode: flyspell
% ispell-local-dictionary: "castellano8"
% mode: latex
% End:
