\chapter{Método de Trabajo}
\label{chap:metodo}

@TODO

\section{Scrum}

Scrum es un marco de trabajo inicialmente ideado para gestionar y desarrollar productos. Pero
se puede encontrar equipos utilizando Scrum en multitud de ámbitos como en desarrollos de 
software, hardware, vehículos, escuelas y en casi cualquier cosa.

Se emplea un desarrollo iterativo e incremental para realizar mejores estimaciiones y controlar 
los riesgos. 

...........


\section{Herramientas}

Para la realización de este \acs{TFG} se han utilizado los medios que se presentan a continuación.

\subsection{Hardware}

Para la realización de este TFG se utilizará un ordenador, un móvil Android y un iPhone. El equipo 
necesario ha sido facilitado en su mayoría por Intelygenz.

\begin{description}
	\item MacBook Pro Procesador Intel Core i5 @2.5GHz. 8GB RAM 1600 MHz. macOS High Sierra v10.13.5
	\item Nexus 5x Procesador Qualcomm⃝R SnapdragonTM 808 64 bits, 1.8 GHz, seis núcleos, 2GB RAM con Android 8.1
	\item Nokia 7 Plus
	\item Iphone 8 Procesador Apple A10 Fusion Quad-Core 64 bits, 2.23 GHz, 2GB RAM con IOS 11
\end{description}

\subsection{Software}

 El software necesario para la realización del proyecto es el siguiente:
\begin{description}
	\item Android Studio~\cite{ASTUDIO}, para desarrollar la parte nativa de Android de la aplicación.
	\item ECMAScript~\cite{ECMA}~\cite{ECMABOOK}, es el lenguaje en el que se realiza el desarrollo.
	\item IntelliJ IDEA Community~\cite{IDEA}, para desarrollar la mayor parte de la aplicación.
	\item React Native~\cite{RENA}~\cite{REACTBOOK}, es el framework que se usa para realizar el desarrollo.
	\item Redux~\cite{REDUX}, para desacoplar el estado global en React Native utilizando el patrón Flux.
	\item Redux Thunk~\cite{THUNK}, es un middleware que se usará para disparar acciones de redux.
	\item Jest~\cite{JEST}, se utilizará para realizar tests unitarios a la aplicación.
	\item Native Base~\cite{NABA}, se utilizará para realizar el estilo visual de la aplicación.
	\item Standard~\cite{STAND}, se usará como guía de estilo del código.
	\item Xcode~\cite{XCODE}, para desarrollar la parte nativa de IOS de la aplicación.
\end{description}

% Local Variables:
%  coding: utf-8
%  mode: latex
%  mode: flyspell
%  ispell-local-dictionary: "castellano8"
% End:
