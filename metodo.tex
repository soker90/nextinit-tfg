\chapter{Método de Trabajo}
\label{chap:metodo}

\drop{E}{n} este capítulo se expone la metodología de trabajo utilizada en el proyecto, Scrum. Además,
se exponen las herramientas utilizadas para elaborar el \acs{TFG}.

\section{Scrum}

Scrum es un marco de trabajo donde se pueden emplear diferentes procesos y técnicas. Inicialmente 
fue ideado para gestionar y desarrollar productos, pero se puede encontrar equipos utilizando Scrum 
en multitud de ámbitos como en desarrollos de software, hardware, vehículos, escuelas y en casi cualquier 
cosa.

Se emplea un desarrollo iterativo e incremental para realizar mejores estimaciones y controlar 
los riesgos. Los equipos Scrum son pequeños, flexibles y adaptativos. 

\subsection{Equipo y roles}

El equipo Scrum lo forman un Product Owner o propietario del producto, un equipo de desarrolladores y el Scrum Master. Algunas
de las características de los equipos Scrum son:

\begin{itemize}
	\item Son auto-organizados y multifuncionales.
	\item Eligen la mejor forma de llevar a cabo su trabajo y no los dirige ninguna persona externa al equipo.
	\item Los equipos tienen todas las competencias y habilidades necesarias para llevar a cabo el trabajo,
	sin necesitar personas externas al equipo.
	\item Están diseñados para optimizar la flexibilidad, la creatividad y la productividad.
	\item Entregan los productos de forma iterativa e incremental, facilitando la retroalimentación. Cada versión entregada
	es funcional y útil para el Product Owner.
\end{itemize}

\subsubsection{Product Owner}

Es el propietario del producto que desarrolla el equipo, es el responsable de trasladar al equipo el punto de vista del cliente y maximizar
el valor del producto que se esta desarrollando. Además, es la única persona responsable de gestionar la Pila del producto (Product Backlog), 
esta gestión incluye:

\begin{itemize}
	\item Expresar de forma clara las historias de usuario.
	\item Priorizar los elementos de la Product Backlog para alcanzar los objetivos.
	\item Mantener la Product Backlog visible, transparente y clara, debe mostrar lo que el equipo hará a continuación.
	\item Asegurar que el equipo entienda los elementos de la Product Backlog al nivel necesario.
\end{itemize}

\subsubsection{Equipo de desarrollo}

El equipo de desarrolladores es un grupo de personas que realizan el trabajo necesario para cumplir con las
necesidades plasmadas en las historias de usuario del Sprint Backlog(historias de usuario del sprint). Al final 
del sprint deben entregar un trabajo terminado que pueda ser puesto en producción.

La organización es la encargada de crear estos equipos de desarrollo para que estos se organicen y gestionen 
su trabajo. Dichos equipos tienen las siguientes características:

\begin{itemize}
	\item Son auto-organizados, nadie lidera al equipo.
	\item Son multifuncionales, con las habilidades necesarias para llevar a cabo cada incremento del producto.
	\item No hay titulos para ningún miembro independientemente del trabajo que realice.
	\item No hay subequipos, tales como equipo de pruebas o arquitectura.
	\item Cada miembro puede tener habilidades especializadas o centrarse más en un área pero la responsabilidad
	del trabajo recae en todo el equipo.
\end{itemize}

El equipo debe ser lo suficientemente pequeño para ser ágil pero lo suficientemente grande como para completar 
un cantidad de trabajo significativa. Tener un numero muy pequeño reduce la interacción entre el equipo y es 
menos productivo. Además, aumenta la probabilidad de no tener las habilidades necesarias para terminar el 
sprint.

\subsubsection{Scrum Master}

El Scrum Master es el responsable de que se utilize Scrum correctamente, ayudan al resto del equipo a cumplir 
la teoría de Scrum, prácticas, reglas y valores. A continuación, se puede ver como ayuda a cada miembro del 
equipo.

Algunas de las formas en las que el Scrum Master ayuda al Product Owner son las siguientes:
\begin{itemize}
	\item Asegura que el equipo de desarrolladores entienda los objetivos, el alcance y el dominio del producto.
	\item Ayudar a gestionar la Product Backlog de manera efectiva.
	\item Ayuda al Equipo Scrum a entender la necesidad de que los elementos la Product Backlog sean claros y 
	concisos
	\item Entender la planificación del producto
	\item Asegurar que sepa como ordenar la Product Backlog
	\item A ser ágil.
	\item Facilitar que se realicen los eventos de Scrum
 \end{itemize}

Algunas de las formas en las que el Scrum Master ayuda al equipo de desarrollo son las siguientes:
\begin{itemize}
	\item Sirve de guía para que sean auto-organizado y multifuncional.
	\item Ayuda a crear productos con valor.
	\item Eliminar cualquier impedimento que pueda entorpecer el progreso del Equipo de Desarrollo.
	\item Facilita que se produzcan los eventos de Scrum que se necesiten.
	\item Guiar al equipo en la organización en los entornos que no se entienda como aplicar Scrum.
\end{itemize}

El Scrum Master tambuen ayuda a la organización de algunas formas como:
\begin{itemize}
	\item La lidera y guía para adoptar Scrum.
	\item Planifica las implementaciones de Scrum en ella.
	\item Ayuda a empleados y clientes utilizar Scrum correctamente.
	\item Propone y lleva a cabo cambios que aumenten la productividad.
\end{itemize}

\subsection{Eventos}



\subsection{Artefactos}

\subsection{Reglas}


\section{Herramientas}

Para la realización de este \acs{TFG} se han utilizado los medios que se presentan a continuación.

\subsection{Hardware}

Para la realización de este TFG se utilizará un ordenador, un móvil Android y un iPhone. El equipo 
necesario ha sido facilitado en su mayoría por Intelygenz.

\begin{itemize}
	\item MacBook Pro Procesador Intel Core i5 @2.5GHz. 8GB RAM 1600 MHz. macOS High Sierra v10.13.6
	\item Nexus 5x Procesador Qualcomm® Snapdragon™ 808 64 bits, 1.8 GHz, seis núcleos, 2GB RAM con Android 8.1
	\item Nokia 7 Plus Procesador Qualcomm® Snapdragon™ 660 64 bits, 2,2 GHz, ocho núcleos, 4GB RAM con Android 9.0
	\item iPhone 8 Procesador Apple A10 Fusion Quad-Core 64 bits, 2.23 GHz, 2GB RAM con IOS 11
\end{itemize}

\subsection{Software}

 El software necesario para la realización del proyecto es el siguiente:
\begin{itemize}
	\item Android Studio~\cite{ASTUDIO}, para desarrollar la parte nativa de Android de la aplicación.
	\item ECMAScript~\cite{ECMA}~\cite{ECMABOOK}, es el lenguaje en el que se realiza el desarrollo.
	\item Genymotion
	\item GitHub
	\item GitKraken
	\item GitLab Community Edition
	\item Invision
	\item IntelliJ IDEA Community~\cite{IDEA}, para desarrollar la mayor parte de la aplicación.
	\item Native Base~\cite{NABA}, se utilizará para realizar el estilo visual de la aplicación.
	\item React Native~\cite{RENA}~\cite{REACTBOOK}, es el framework que se usa para realizar el desarrollo.
	\item Redux~\cite{REDUX}, para desacoplar el estado global en React Native utilizando el patrón Flux.
	\item Slack
	\item TeXstudio
	\item Travis
	\item Xcode~\cite{XCODE}, para desarrollar la parte nativa de IOS de la aplicación.
\end{itemize}

% Local Variables:
%  coding: utf-8
%  mode: latex
%  mode: flyspell
%  ispell-local-dictionary: "castellano8"
% End:
