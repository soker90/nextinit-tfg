\chapter{Método de Trabajo}
\label{chap:metodo}

\drop{E}{n} este capítulo se expone la metodología de trabajo utilizada en el proyecto, Scrum. Además,
se exponen las herramientas utilizadas para elaborar el \acs{TFG}.

\section{Scrum}

Scrum es un marco de trabajo donde se pueden emplear diferentes procesos y técnicas. Inicialmente 
fue ideado para gestionar y desarrollar productos, pero se puede encontrar equipos utilizando Scrum 
en multitud de ámbitos como en desarrollos de software, hardware, vehículos, escuelas y en casi cualquier 
cosa.

Se emplea un desarrollo iterativo e incremental para realizar mejores estimaciones y controlar 
los riesgos. Los equipos Scrum son pequeños, flexibles y adaptativos. 

\subsection{Equipo}

El equipo Scrum lo forman un Product Owner o propietario del producto, un equipo de desarrolladores y el Scrum Master. Algunas
de las características de los equipos Scrum son:

\begin{itemize}
	\item Son auto-organizados y multifuncionales.
	\item Eligen la mejor forma de llevar a cabo su trabajo y no los dirige ninguna persona externa al equipo.
	\item Los equipos tienen todas las competencias y habilidades necesarias para llevar a cabo el trabajo,
	sin necesitar personas externas al equipo.
	\item Están diseñados para optimizar la flexibilidad, la creatividad y la productividad.
	\item Entregan los productos de forma iterativa e incremental, facilitando la retroalimentación. Cada versión entregada
	es funcional y útil para el Product Owner.
\end{itemize}

\subsubsection{Product Owner}

Es el propietario del producto que desarrolla el equipo, es el responsable de trasladar al equipo el punto de vista del cliente y maximizar
el valor del producto que se esta desarrollando. Además, es la única persona responsable de gestionar la Pila del producto (Product Backlog), 
esta gestión incluye:

\begin{itemsize}
	\item Expresar de forma clara las historias de usuario.
	\item Priorizar los elementos de la Product Backlog para alcanzar los objetivos.
	\item Mantener la Product Backlog visible, transparente y clara, debe mostrar lo que el equipo hará a continuación.
	\item Asegurar que el equipo entienda los elementos de la Product Backlog al nivel necesario.
\end{itemsize}

\subsubsection{Equipo de desarrollo}



\subsubsection{Scrum Master}

\subsubsection{Product Owner}


\subsection{Roles}

\subsection{Eventos}

\subsection{Artefactos}

\subsection{Reglas}

...........


\section{Herramientas}

Para la realización de este \acs{TFG} se han utilizado los medios que se presentan a continuación.

\subsection{Hardware}

Para la realización de este TFG se utilizará un ordenador, un móvil Android y un iPhone. El equipo 
necesario ha sido facilitado en su mayoría por Intelygenz.

\begin{itemize}
	\item MacBook Pro Procesador Intel Core i5 @2.5GHz. 8GB RAM 1600 MHz. macOS High Sierra v10.13.6
	\item Nexus 5x Procesador Qualcomm® Snapdragon™ 808 64 bits, 1.8 GHz, seis núcleos, 2GB RAM con Android 8.1
	\item Nokia 7 Plus Procesador Qualcomm® Snapdragon™ 660 64 bits, 2,2 GHz, ocho núcleos, 4GB RAM con Android 9.0
	\item iPhone 8 Procesador Apple A10 Fusion Quad-Core 64 bits, 2.23 GHz, 2GB RAM con IOS 11
\end{itemize}

\subsection{Software}

 El software necesario para la realización del proyecto es el siguiente:
\begin{itemize}
	\item Android Studio~\cite{ASTUDIO}, para desarrollar la parte nativa de Android de la aplicación.
	\item ECMAScript~\cite{ECMA}~\cite{ECMABOOK}, es el lenguaje en el que se realiza el desarrollo.
	\item Genymotion
	\item GitHub
	\item GitKraken
	\item GitLab Community Edition
	\item Invision
	\item IntelliJ IDEA Community~\cite{IDEA}, para desarrollar la mayor parte de la aplicación.
	\item Native Base~\cite{NABA}, se utilizará para realizar el estilo visual de la aplicación.
	\item React Native~\cite{RENA}~\cite{REACTBOOK}, es el framework que se usa para realizar el desarrollo.
	\item Redux~\cite{REDUX}, para desacoplar el estado global en React Native utilizando el patrón Flux.
	\item Slack
	\item TeXstudio
	\item Travis
	\item Xcode~\cite{XCODE}, para desarrollar la parte nativa de IOS de la aplicación.
\end{itemize}

% Local Variables:
%  coding: utf-8
%  mode: latex
%  mode: flyspell
%  ispell-local-dictionary: "castellano8"
% End:
