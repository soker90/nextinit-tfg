\chapter{Objetivos}
\label{chap:objetivos}

\noindent
De acuerdo a la Introducción, el alumno deberá especificar cuál(es) es(son)
la(s) hipótesis de trabajo de la(s) que se parte(n), qué se pretende(n) resolver
en el presente TFM.

Es importante formular con claridad cuál es el objetivo general y el alcance
correspondiente a las hipótesis de trabajo. Del mismo modo, se deberán
establecer los objetivos parciales derivados del objetivo general y los
resultados esperados.

Como preámbulo a la formulación de los objetivos parciales, el alumno deberá
discutir sobre las limitaciones y condicionantes a tener en cuenta en el
desarrollo del TFM (lenguaje de desarrollo, equipos, madurez de la tecnología,
etcétera).

\emph{A modo de referencia, este capítulo tendrá una longitud aproximada de
  entre 2 y 10 páginas}.

\section{Objetivo general}

El hito final que se pretende lograr, destacando el dominio concreto estudiado,
el problema específico que resuelve y/o la funcionalidad que se aporta en el
presente trabajo.


\section{Objetivos específicos}

Los objetivos específicos son las partes independientes del proyecto que tienen
valor por sí mismas.

Por ejemplo, si el objetivo general fuera destruir una flota enemiga, los
objetivos específicos podrían ser: hundir el portaaviones, inutilizar las
torretas de los destructores, eliminar los cazas enemigos, etc.

Los objetivos específicos no son tareas; análisis, diseño, etc. no tienen valor
intrínseco para el cliente, si por ejemplo el proyecto se cancela en la fase de
diseño no se le entrega nada de valor al cliente, luego no se cubre ningún
objetivo.

No se deben confundir los objetivos del proyecto con los objetivos del
alumno. Indicar como objetivo que el alumno va a aprender o a estudiar
determinada disciplina o herramienta no aporta nada al cliente. Deben ser
entregables que el cliente puede valorar y por los que estaría dispuesto
a pagar. Resumiendo, son \textbf{objetivos}, no subjetivos.

\subsection{Objetivo 1}

\subsection{Objetivo 2}

\subsection{Objetivo 3}


% Local Variables:
%  coding: utf-8
%  mode: latex
%  mode: flyspell
%  ispell-local-dictionary: "castellano8"
% End:
