\chapter{Introducción}

\drop{E}{n} la actualidad, cada vez es más frecuente escuchar hablar del talento humano de las empresas. Éste es el mejor recurso de estas,
 los empleados son los que mejor conocen su trabajo y pueden tener ideas de como mejorar diversos aspectos de la empresa. 
 
 Los clásicos buzones de sugerencias pueden servir a las empresas para conocer las mayores molestias de algunos de sus empleados. Estos 
 rápidamente pueden convertirse en buzones de quejas, o los empleados pueden sentir que no se toman en cuenta sus opiniones si estas 
 sugerencias o quejas no obtienen una respuesta.
 
 Un sistema de administración de ideas pueden ayudan a las empresas a fomentar la innovación desde cualquiera de los miembros de estas. Los 
 principales beneficios de estos sistemas son:
 
 \begin{itemize}
 	\item Aumento del compromiso de los empleados.
 	\item Promueve la transparencia dentro de la empresa.
 	\item Permite a los empleados dar su opinión.
 	\item Detecta y retiene el talento.
 	\item Mejora el clima laboral.
 	\item Alinea todos los empleados con los objetivos estratégicos de la compañía.
 	\item Genera ideas apoyadas por los empleados para mejorar la empresa en ámbitos muy diversos.
 \end{itemize}
 
 
 \section{Nextinit}
 
 Nextinit es un sistema de gestión de ideas


Talento humano, el mayor recurso de las empresas es la gente, 
las empresas buscan mejoras en el modelos de negocio, mejorar la comunicación con los empleados, buzón de sugerencias ponen los que mas les molesta
El capítulo de introducción debe dar al lector una perspectiva básica ---pero
completa--- del problema que se pretende abordar, pero también de la estrategia
y enfoque que el autor propone para su resolución. El lector debería poder
determinar si este documento le interesa leyendo únicamente este capítulo.


\section{Estructura del documento}

Pueden incluirse aquí una sección con algunos consejos para la lectura del
documento dependiendo de la motivación o conocimientos del lector.  También
puede ser útil incluir una lista con el nombre y finalidad de cada uno de los
capítulos restantes.

\begin{definitionlist}
\item[Capítulo \ref{chap:objetivos}: \nameref{chap:objetivos}] Finalidad y justificación
  (con todo detalle) del presente documento.
 \item[Capítulo \ref{chap:antecedentes}: \nameref{chap:antecedentes}] Explica herramientas
  y aspectos básicos de edición con \LaTeX.
  \item [Capítulo \ref{chap:metodo}: \nameref{chap:metodo}] El método de trabajo utilizado
  \item[Capítulo \ref{chap:resultados}: \nameref{chap:resultados}] Resultados del TFG
  \item[Capítulo \ref{chap:conclusiones}: \nameref{chap:conclusiones}] Las conclusiones van aquí.
\end{definitionlist}

% Local Variables:
%  coding: utf-8
%  mode: latex
%  mode: flyspell
%  ispell-local-dictionary: "castellano8"
% End: