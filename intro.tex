\chapter{Introducción}

\drop{E}{n} la actualidad, cada vez es más frecuente escuchar hablar del talento humano de las empresas. Éste es el mejor recurso de estas,
 los empleados son los que mejor conocen su trabajo y pueden tener ideas de como mejorar diversos aspectos de la empresa. 
 
 Los clásicos buzones de sugerencias pueden servir a las empresas para conocer las mayores molestias de algunos de sus empleados. Estos 
 rápidamente pueden convertirse en buzones de quejas, o los empleados pueden sentir que no se toman en cuenta sus opiniones si estas 
 sugerencias o quejas no obtienen una respuesta.
 
 Un sistema de administración de ideas pueden ayudan a las empresas a fomentar la innovación desde cualquiera de los miembros de estas. Los 
 principales beneficios de estos sistemas son:
 
 \begin{itemize}
 	\item Aumento del compromiso de los empleados.
 	\item Promueve la transparencia dentro de la empresa.
 	\item Permite a los empleados dar su opinión.
 	\item Detecta y retiene el talento.
 	\item Mejora el clima laboral.
 	\item Alinea todos los empleados con los objetivos estratégicos de la compañía.
 	\item Genera ideas apoyadas por los empleados para mejorar la empresa en ámbitos muy diversos.
 \end{itemize}
 
 
 \section{Nextinit}
 
 Nextinit es un sistema de gestión de ideas, una herramienta que permite que cualquier miembro de una organización pueda proponer ideas para
 mejorarla y resolver desafíos lanzados por esta. Las ideas se crean dentro de la plataforma de forma similar a una startup que busca dinero. Los 
 usuarios no solo pueden  proponer ideas, sino invertir dinero virtual para que las ideas propuestas en la plataforma aumenten el capital invertido 
 y puedan salir adelante.
 
 La idea de esta plataforma es usar la inteligencia colectiva para mejorar las organizaciones. Como el dinero virtual de cada usuario es limitado, estos
  solo invertirán,  en las mejores ideas, por lo que si mucha gente invierte en una idea, será porque dicha idea es buena. Y si esa idea se lleva a cabo 
  se recupera el dinero invertido  mas unos intereses.
 
\section{Estructura del documento}

El presente documento se compone con la siguiente estructura:

\begin{definitionlist}
\item[Capítulo \ref{chap:objetivos}: \nameref{chap:objetivos}] Se expone el objetivo principal y sus subjetivos.
 \item[Capítulo \ref{chap:antecedentes}: \nameref{chap:antecedentes}] Se describen los sistemas tomados como referencia utilizados para 
 el desarrollo de este \acs{TFG}.
  \item [Capítulo \ref{chap:metodo}: \nameref{chap:metodo}] Se describe y justifica la metodología utilizada para el desarrollo de este \acs{TFG}.
  \item[Capítulo \ref{chap:resultados}: \nameref{chap:resultados}] Se muestran los resultado obtenidos al aplicar la metodología elegida.
  \item[Capítulo \ref{chap:conclusiones}: \nameref{chap:conclusiones}] Se exponen las conclusiones llegadas al finalizar el desarrollo.
\end{definitionlist}

% Local Variables:
%  coding: utf-8
%  mode: latex
%  mode: flyspell
%  ispell-local-dictionary: "castellano8"
% End