\chapter{Resultados}
\label{chap:resultados}

Este capítulo incluye unas nociones básicas de \LaTeX{} y algunos consejos
sencillos de composición para sacar todo el jugo a la clase \esitfg. Ten
presente que este capítulo está pensado para que leas el código fuente y lo
compares con el resultado en PDF.

\section{Sprint 0}

Debido a su continuo uso, se muestra entre paréntesis la combinación del modo
\texttt{auctex} de GNU Emacs para incluir el comando \LaTeX{} correspondiente.

\section{Sprint 1}

Debido a su continuo uso, se muestra entre paréntesis la combinación del modo
\texttt{auctex} de GNU Emacs para incluir el comando \LaTeX{} correspondiente.

\section{Sprint 2}

Debido a su continuo uso, se muestra entre paréntesis la combinación del modo
\texttt{auctex} de GNU Emacs para incluir el comando \LaTeX{} correspondiente.

\section{Sprint 3}

Debido a su continuo uso, se muestra entre paréntesis la combinación del modo
\texttt{auctex} de GNU Emacs para incluir el comando \LaTeX{} correspondiente.

\section{Sprint 4}

Debido a su continuo uso, se muestra entre paréntesis la combinación del modo
\texttt{auctex} de GNU Emacs para incluir el comando \LaTeX{} correspondiente.

\section{Sprint 5}

Debido a su continuo uso, se muestra entre paréntesis la combinación del modo
\texttt{auctex} de GNU Emacs para incluir el comando \LaTeX{} correspondiente.

% Local Variables:
%  coding: utf-8
%  mode: latex
%  mode: flyspell
%  ispell-local-dictionary: "castellano8"
% End:
