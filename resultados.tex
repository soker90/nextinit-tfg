\chapter{Resultados}
\label{chap:resultados}

\drop{E}{n} este capitulo de detalla el proceso y los pasos dados para la realización del 
proyecto. Se ha divido este apartado en dos partes puesto que realmente se 
desarrollaron dos proyectos, uno dependiente de los resultado del otro. 

\section{Análisis y viabilidad de las tecnologías}

Antes de desarrollar la aplicación final, la organización quería que se desarrollara una 
aplicación de pruebas que utilizará datos falsos y que se simularan las llamadas a las APIs. Con 
esto se quería comprobar el rendimiento de una aplicación con React Native, la fluidez, las 
transiciones entre pantallas, las animaciones y probar las posibles librerías que se 
podrían utilizar en la versión final.

\subsection{Sprint 0}

En esta primera fase se instaló todas las herramientas necesarias para llevar acabo el 
proyecto. Se crearon los repositorios, uno en GitLab para la aplicación que se iba 
a crear y otro en GitHub para alojar el código fuente de la memoria. También se 
preparó Travis para que generara el pdf del anteproyecto y la memoria.

Después se creó un proyecto de React Native y se configuró en entorno y  las librerías 
necesarias como redux.

\subsection{Sprint 1}

En este primer Sprint se realizó la pantalla de bienvenida y otra para el inicio de sesión junto 
con la funcionalidad para iniciar sesión a través de la API de Nextinit en un entorno de 
pruebas y las animaciones requeridas.

También se creo otra pantalla que permite al usuario una vez ha iniciado sesión,
elegir en cuál de los Nextinit de los que dispone quiere entrar.

Además se preparó la navegación a través de la aplicación y se realizaron diversos 
cambios estéticos.

\subsection{Sprint 2}

Se creó la pantalla de

\subsection{Sprint 3}



Herramientas
- librerias

\section{Desarrollo de la aplicación}

\subsection{Sprint 0}

Debido a su continuo uso, se muestra entre paréntesis la combinación del modo
\texttt{auctex} de GNU Emacs para incluir el comando \LaTeX{} correspondiente.

\subsection{Sprint 1}

Debido a su continuo uso, se muestra entre paréntesis la combinación del modo
\texttt{auctex} de GNU Emacs para incluir el comando \LaTeX{} correspondiente.

\subsection{Sprint 2}

Debido a su continuo uso, se muestra entre paréntesis la combinación del modo
\texttt{auctex} de GNU Emacs para incluir el comando \LaTeX{} correspondiente.

\subsection{Sprint 3}

Debido a su continuo uso, se muestra entre paréntesis la combinación del modo
\texttt{auctex} de GNU Emacs para incluir el comando \LaTeX{} correspondiente.

\subsection{Sprint 4}

Debido a su continuo uso, se muestra entre paréntesis la combinación del modo
\texttt{auctex} de GNU Emacs para incluir el comando \LaTeX{} correspondiente.

\subsection{Sprint 5}

Debido a su continuo uso, se muestra entre paréntesis la combinación del modo
\texttt{auctex} de GNU Emacs para incluir el comando \LaTeX{} correspondiente.

% Local Variables:
%  coding: utf-8
%  mode: latex
%  mode: flyspell
%  ispell-local-dictionary: "castellano8"
% End:
