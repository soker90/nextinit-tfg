\chapter{Resultados}
\label{chap:resultados}

\drop{E}{n} este capitulo de detalla el proceso y los pasos dados para la realización del 
proyecto. Se ha divido este apartado en dos partes puesto que realmente se 
desarrollaron dos proyectos, uno dependiente de los resultado del otro. 

\section{Análisis y viabilidad de las tecnologías}

Antes de desarrollar la aplicación final, la organización quería que se desarrollara una 
aplicación de pruebas que utilizará datos falsos y que se simularan las llamadas a las APIs. Con 
esto se quería comprobar el rendimiento de una aplicación con React Native, la fluidez, las 
transiciones entre pantallas, las animaciones y probar las posibles librerías que se 
podrían utilizar en la versión final.

\subsection{Sprint 0}

En esta primera fase se instaló todas las herramientas necesarias para llevar acabo el 
proyecto. Se crearon los repositorios, uno en GitLab para la aplicación que se iba 
a crear y otro en GitHub para alojar el código fuente de la memoria. También se 
preparó Travis para que generara el pdf del anteproyecto y la memoria.

Después se creó un proyecto de React Native y se configuró en entorno y  las librerías 
necesarias como redux.

\subsection{Sprint 1}

En este primer Sprint se realizó la pantalla de bienvenida y otra para el inicio de sesión junto 
con la funcionalidad para iniciar sesión a través de la API de Nextinit en un entorno de 
pruebas y las animaciones requeridas.

También se creo otra pantalla que permite al usuario una vez ha iniciado sesión,
elegir en cuál de los Nextinit de los que dispone quiere entrar.

Además se preparó la navegación a través de la aplicación y se realizaron diversos 
cambios estéticos.

Se creó la pantalla inicial, a la que entra el usuario después de iniciar sesión y elegir el 
Nextinit. En esta pantalla se muestra una lista con todas las ideas con un scroll infinito
y un carrusel con los desafíos del Nextinit.

Se añadió una serie de pantallas que se muestran la primera vez que se inicia sesión 
en el Nextinit, son la pantalla con información de la RGPD, otra con el texto legal y 
un pequeño tutorial de la aplicación.

Se ha creado un menú lateral que permite acceder a las distintas secciones de la 
aplicación.

Se creó una pantalla a la que se accede desde un botón en la pantalla inicial,
en la que hay dos pestañas, una para ver las notificaciones del usuario y otra para
ver la actividad del Nextinit.


\subsection{Sprint 2}

En este Sprint se creo la pantalla que muestra la información de la idea seleccionada. Se creó 
una nueva pantalla accesible desde el menú lateral que muestra una lista con todas las ideas 
del usuario en las que se indica cuales están publicadas y las que son borradores. 

También se realizó la pantalla con un formulario para crear una nueva idea y una vista previa 
de la idea, en la que se muestra la misma pantalla que cuando se selecciona una idea pero con 
los datos recién introducidos.

Desde la vista previa de la idea se puede volver a la pantalla de edición de la idea o publicarla. En 
caso de publicarla, se da la posibilidad al usuario de realizar la primera inversión.

Al ver la información de la idea, si está buscando fondo aparece un botón que permite al usuario 
acceder al formulario para invertir en esa idea.

Se creó una pantalla accesible desde el menú lateral que muestra una lista con todos los desafíos 
disponibles

Se creó otra pantalla también accesible desde el menú lateral que muestra 4 pestañas en las que se 
pueden ver los rankings de usuarios que más dinero han invertido, que en más ideas han invertido, 
que más ideas han publicado y que tienen más dinero.

Se realizaron diversas pruebas para comprobar el desempeño de la aplicación en diversos idiomas como 
el francés, el chino y el ruso.

Se diseño una pantalla anterior a la de inicio de sesión con varias animaciones con el logo de Nextinit.

Herramientas
- librerias

\section{Desarrollo de la aplicación}

\subsection{Sprint 0}

En desarrollo...

\subsection{Sprint 1}

En desarrollo...

\subsection{Sprint 2}

En desarrollo...

\subsection{Sprint 3}

En desarrollo...

\subsection{Sprint 4}

En desarrollo...

\subsection{Sprint 5}

En desarrollo...

% Local Variables:
%  coding: utf-8
%  mode: latex
%  mode: flyspell
%  ispell-local-dictionary: "castellano8"
% End:
