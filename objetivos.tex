\chapter{Objetivos}
\label{chap:objetivos}

\drop{E}{n} este capitulo se presentan y detallan cada uno de los objetivos y subobjetivos
del presente \afs{TFG}.

\section{Objetivo general}

El objetivo final es construir una aplicación cliente eficiente y funcional para Android e IOS
que permita utilizar la herramienta Nextinit en dichas plataformas.

La aplicación móvil será creada desde cero, se integrará mediante una API REST con la aplicación 
web de Nextinit. Y se incluirá la mayor parte de la funcionalidad existente en las antiguas 
aplicaciones de Nextinit, además, de otras nuevas solicitadas por los clientes.


\section{Objetivos específicos}

En este objetivo general se integran varios subobjetivos:

\subsection{Objetivo 1: Generar ideas y compartir ideas entre los empleados}

El usuario comparte en la plataforma su idea, concebida con el fin de mejorar la empresa o corregir 
inconvenientes detectados en ella. Estas ideas son automáticamente compartidas con el resto de empleados
de la organización.

Cada idea es lanzada en Nextinit y para convertirse en realidad necesita de financiación. Por
ello, los usuarios, además de crear sus propias ideas, se convierten en inversores de crowdfunding,
gracias a una cantidad de dinero virtual con la que todo el mundo empieza. Como inversores,
según la fase del proyecto en la que inviertan, recuperarán más o menos capital. Como
innovadores, se juegan el que su idea pueda convertirse en realidad.
La inteligencia colectiva está en la base de nextinit; entre todos somos más eficientes que una
sola persona, por muy especializada que esté, para detectar el talento y las buenas ideas.
Por ello, nextinit es una cuenta donde cada uno, desde su conocimiento, experiencia, y
percepción, ingresa sus ideas, que podrían generar excelentes soluciones de todo tipo para
afrontar los cambios del día a día, y del futuro.


\subsection{Objetivo 1: Poder compartir ideas entre los empleados}

Mediante la aplicación se debe poder acceder a las ideas de la organización a la que corresponda el usuario. Lo que 
incluye poder publicar ideas y que estas sean vistas y comentadas por parte de cualquier usuario de la aplicación, además, 
cualquier usuario podrá invertir dinero ficticio en ellas.

\subsection{Objetivo 2: Poder lanzar desafíos que fomenten la creación de ideas}
Los desafíos exponen un problema de la organización, dentro de el los usuarios pueden crear ideas para
proponer las soluciones o mejoras a ese problema.

\subsection{Objetivo 3: Aumentar el compromiso de los empleados}




% Local Variables:
%  coding: utf-8
%  mode: latex
%  mode: flyspell
%  ispell-local-dictionary: "castellano8"
% End:
