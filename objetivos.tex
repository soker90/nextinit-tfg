\chapter{Objetivos}
\label{chap:objetivos}

\drop{E}{n} este capitulo se presentan y detallan cada uno de los objetivos y subobjetivos
del presente \afs{TFG}.

\section{Objetivo general}

El objetivo final es construir una aplicación cliente eficiente y funcional para Android e IOS
que permita utilizar la herramienta Nextinit en dichas plataformas.

La aplicación móvil será creada desde cero, se integrará mediante una API REST con la aplicación 
web de Nextinit. Y se incluirá la mayor parte de la funcionalidad existente en las antiguas 
aplicaciones de Nextinit, además, de otras nuevas solicitadas por los clientes.


\section{Objetivos específicos}

En este objetivo general se integran varios subobjetivos:

\subsection{Objetivo 1: Evaluar la viabilidad de desarrollar la aplicación con tecnologías híbridas}

Antes de decidir si hacer la aplicación de Nextinit con tecnologías híbridas como React Native, Intelygenz quiere ver
una aplicación sencilla de ejemplo para poder comprobar la fluidez, las transiciones entre las distintas pantallas y las
animaciones, además de buscar y probar las librerías que se utilizarán en el proyecto final.

\subsection{Objetivo 2: Poder compartir ideas entre los empleados}

Mediante la aplicación se debe poder acceder a las ideas de la organización a la que corresponda el usuario. Lo que 
incluye poder publicar ideas y que estas sean vistas y comentadas por parte de cualquier usuario de la aplicación, además, 
cualquier usuario podrá invertir dinero ficticio en ellas.

\subsection{Objetivo 3: Poder lanzar desafíos que fomenten la creación de ideas}
Los desafíos exponen un problema de la organización, dentro de el los usuarios pueden crear ideas para
proponer las soluciones o mejoras a ese problema.


% Local Variables:
%  coding: utf-8
%  mode: latex
%  mode: flyspell
%  ispell-local-dictionary: "castellano8"
% End:
