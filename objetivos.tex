\chapter{Objetivos}
\label{chap:objetivos}

En este capitulo se  presentan y detallan cada uno de los objetivos y subjetivos
del presente \afs{TFG}.
\section{Objetivo general}

El objetivo final es construir una aplicación funcional para móviles que permita
utilizar Nextinit en Android e IOS.


\section{Objetivos específicos}

En este objetivo general se integran varios subjetivos 



Los objetivos específicos (o sub-objetivos) son las partes independientes del
proyecto que tienen valor para el cliente por si mismas. Es habitual confundir
los objetivos específicos con requisitos, fases o tareas. Para aclarar la
diferencia pongo aquí un ejemplo.

Supón que el objetivo general es destruir una flota enemiga. Los
objetivos específicos podrían ser: hundir el portaaviones, inutilizar las
torretas de los destructores, eliminar los cazas enemigos, etc.

Los sub-objetivos \textbf{no son tareas ni fases}.  En nuestro ejemplo, las
tareas podrían ser: determinar las fuerzas de la flota enemiga, elegir el
armamento más adecuado, hacer la dotación de los buques propios. Es importante
tener claro que las tareas no tienen valor intrínseco para el cliente. Por
ejemplo, si en un proyecto con desarrollo en cascada se cancela en la fase de
diseño, no se le entrega nada de valor al cliente, es decir, el análisis
realizado no cubre ningún sub-objetivo.

Los sub-objetivos \textbf{no son requisitos}. Los requisitos son concrecciones o
limitaciones que determina el cliente. En nuestro ejemplo bélico,algunos
requisitos podrían ser: eliminar primero los cazas enemigos, utilizar únicamente
explosivos convencionales.

Tampoco se deben confundir los objetivos del proyecto con los objetivos del
alumno. Indicar como objetivo que el alumno va a aprender o a estudiar
determinada disciplina o herramienta no aporta nada al cliente. Deben ser
entregables que el cliente puede valorar y por los que estaría dispuesto a
pagar. Resumiendo, tienen que ser \textbf{objetivos}, no subjetivos.

\subsection{Objetivo 1}

\subsection{Objetivo 2}

\subsection{Objetivo 3}


% Local Variables:
%  coding: utf-8
%  mode: latex
%  mode: flyspell
%  ispell-local-dictionary: "castellano8"
% End:
