\chapter{Objetivos}
\label{chap:objetivos}

\drop{E}{n} este capitulo se presentan y detallan cada uno de los objetivos y subobjetivos
del presente \afs{TFG}.

\section{Objetivo general}

El objetivo final es construir una aplicación cliente eficiente y funcional para Android e IOS
que permita utilizar la herramienta Nextinit en dichas plataformas.

La aplicación móvil será creada desde cero, se integrará mediante una API REST con la aplicación 
web de Nextinit. Y se incluirá la mayor parte de la funcionalidad existente en las antiguas 
aplicaciones de Nextinit, además, de otras nuevas solicitadas por los clientes.


\section{Objetivos específicos}

En este objetivo general se integran varios subobjetivos:

\subsection{Objetivo 1: Generar y compartir ideas entre los empleados}

El usuario comparte en la plataforma su idea, concebida con el fin de mejorar la empresa o corregir 
inconvenientes detectados en ella. Estas ideas son automáticamente compartidas con el resto de empleados
de la organización.

Cada idea es lanzada en Nextinit y para convertirse en realidad necesita de financiación. Por
ello, los usuarios, además de crear sus propias ideas, se convierten en inversores de crowdfunding,
gracias a una cantidad de dinero virtual con la que todo el mundo empieza. Como inversores,
según la fase del proyecto en la que inviertan, recuperarán más o menos capital. Como
innovadores, se juegan el que su idea pueda convertirse en realidad.

La inteligencia colectiva es la base de nextinit, entre todos los usuarios son mas eficientes que un 
reducido grupo de innovación, por muy especializados que estén.

Por ese motivo nextinit es una herramienta donde cada usuario, con su conocimiento, experiencia, y
visión, comparte ideas que podrían generar grandes soluciones para mejorar el futuro de la organización.

\subsection{Objetivo 2: Poder lanzar desafíos que fomenten la creación de ideas}
Los desafíos exponen un problema de la organización, dentro de él los usuarios pueden crear ideas para
proponer las soluciones o mejoras a dicho problema. De esta manera, de un mismo problema se obtienen múltiples
 soluciones. Así, entre las diferentes ideas propuestas por los usuarios destacarán las mejores valoradas de
 entre las cuales la organización podrá elegir cuales implementar.
 
Se pueden lanzar desafíos para pensar en un nuevo producto, abordar mejoras en los ya existentes, mejorar
 algún aspecto de la organización o para cualquier otra cosa que desee la organización.


\subsection{Objetivo 3: Aumentar el compromiso de los miembros de la organización}

Al lanzar ideas, los miembros de la organización participan e influyen directamente en la toma de 
decisiones y la dirección de la organización. Este hecho hace a los miembros de la organización sentirse
mas involucrados en la organización y aumentar así su compromiso con esta.


% Local Variables:
%  coding: utf-8
%  mode: latex
%  mode: flyspell
%  ispell-local-dictionary: "castellano8"
% End:
